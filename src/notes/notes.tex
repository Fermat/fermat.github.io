\documentclass[10pt]{article}
\usepackage[margin=3cm]{geometry}
\usepackage{hyperref}
\title{\bfseries\Large Notes and Articles}
%\author{peng-fu@uiowa.edu}
\date{}
\begin{document}
\maketitle
\vspace{-4em}

\vspace{20pt}

\section*{Notes}
\begin{itemize}
\item N001: Reducibility Method for Call-by-Value Simply Typed Lambda Calculus, March 2010. \href{here}{[PDF]}

\item N002: Subtyping Relation as Reducibility Set Inclusion, March 2010. [PDF]

\item N003: Reducibility Method for Call-by-Value System F, June 2010. [PDF]

\item N004: Type Preservation for System F \`a la Curry, 2011. [PDF]

\item N005: A Novel Rewriting Approach to System F \`a la Curry, 2011. [PDF]

\item N006: A Study Note on Quantified Modal Logic, 2011. [PDF]

\item N007: Confluence of Lambda Calculus Modulo Mu Equivalence. Nov 2012. [PDF]

\noindent  \textbf{Remarks}: It turns out this note contain an error similar to the one pointed out in Barendregt's Lambda book's exercise 11.5.4. More specifically, the $t_3$ in lemma 8 can not be found in some situation. 

\item N008: On Program Verification and Language Design, Nov 2012. [PDF]

\item N009: Confluence for Local Lambda-Mu Calculus, July 2013. [PDF]

\noindent \textbf{Remarks}: This is the revised version to deal with the confluence problem arise in N007, the proof is inspired by Thérèse Hardin's interpretation method. 

\item N010: Type Preservation for Curry Style $\mathbf{F}_{\omega}$, July 2013. [PDF]

\noindent \textbf{Remarks}: This note straightforwardly extends Barendregt's method of proving type preservation for Curry style system F(N004) to $\mathbf{F}_{\omega}$. 


  
\end{itemize}

\section*{Articles}

\begin{itemize}
\item A001: A Rewriting Formulation of Curry Style System F. [PDF]

\noindent Peng Fu, Aaron Stump. Jan 2012.

\item A002: Lambda Encoding, Types and Confluence. [PDF]

\noindent Peng Fu. May 2013.

\noindent \textbf{Remarks}: This is my comprehensive exam paper. 


\end{itemize}
\end{document}
