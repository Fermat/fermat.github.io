\documentclass[10pt]{article}
\usepackage[margin=3cm]{geometry}
\usepackage{hyperref}
\usepackage{amsmath}
\usepackage{amssymb} 
\usepackage{csquotes}
\title{\bfseries\Large Notes and Articles}
%\author{peng-fu@uiowa.edu}
\date{}
\begin{document}
\maketitle
\vspace{-4em}

\vspace{20pt}


\section*{Notes}
\begin{itemize}
\item N001: Reducibility Method for Call-by-Value Simply Typed Lambda Calculus, March 2010. [\href{../../document/notes/reducibility-stlc.pdf}{pdf}]

\item N002: Subtyping Relation as Reducibility Set Inclusion, March 2010. [\href{../../document/notes/subtyping-st.pdf}{pdf}]

\item N003: Reducibility Method for Call-by-Value System F, June 2010. [\href{../../document/notes/system-f-iii.pdf}{pdf}]

\item N004: Type Preservation for System F a la Curry, 2011. [\href{../../document/notes/systemf.pdf}{pdf}] 

\item N005: A Novel Rewriting Approach to System F a la Curry, 2011. [\href{../../document/notes/side-proj.pdf}{pdf}] 

\item N006: A Study Note on Quantified Modal Logic, 2011. [\href{../../document/notes/kripke-models.pdf}{pdf}]

\item N007: Confluence of Lambda Calculus Modulo Mu Equivalence. Nov 2012. [\href{../../document/notes/conf-lambda-mu-I.pdf}{pdf}]

\noindent  \textbf{Remarks}: It turns out this note contain an error similar to the one pointed out in exercise 11.5.4 in the Barendregt-lambda-book. More specifically, the $t_3$ in lemma 8 can not be found in some situation. 

%%\item N008: On Program Verification and Language Design, Nov 2012. [\href{here}{PDF}]

\item N008: Confluence for Local Lambda-Mu Calculus, July 2013. [\href{../../document/notes/conf-lambda-mu-II.pdf}{pdf}]

\noindent \textbf{Remarks}: This is the revised version to deal with the confluence problem arise in N007, the proof is inspired by the interpretation method. 

\item N009: Type Preservation for Curry Style F-omega, July 2013. [\href{../../document/notes/fomega-presv.pdf}{pdf}]

\noindent \textbf{Remarks}: This note straightforwardly extends the method(by Barendregt) of proving type preservation for Curry style system F (N004) to F-omega

\item N010: Type Class and Logic Programming with Term Matching, Nov 2014. [\href{../../document/notes/tm.pdf}{pdf}]

\noindent \textbf{Remarks}: This note shows how one can have a notion of \textit{logic programming} based solely on term matching. 

\item N011: Realizability at work,  June, 2017. [\href{../../document/notes/realization.pdf}{pdf}]

  \noindent \textbf{Remarks}: This notes shows examples of realization by termination and realization by shape index. 

\item N012: Shape Realization,  August, 2017. [\href{../../document/notes/shape-real.pdf}{pdf}]

  \noindent \textbf{Remarks}: This notes shows how to obtain a shape indexed version of a linear language. However, it suffers from an error due to the definition of shape on term. More specifically, the variable case for Theorem 5 does not hold. A more specialized version of shape will
  have to define for Theorem 5. This error was pointed out by Peter Selinger.
  

  
\end{itemize}

\section*{Articles}

\begin{itemize}

\item A001: A Rewriting Formulation of Curry Style System F. [\href{../../document/notes/rewrite-f.pdf}{pdf}]

\noindent Peng Fu, Aaron Stump. Jan 2012.

\item A002: Lambda Encoding, Types and Confluence. [\href{../../document/notes/comp-exam.pdf}{pdf}]

\noindent Peng Fu. May 2013.

\noindent \textbf{Remarks}: This is my comprehensive exam paper. 

\item A003: Lambda Encoding with Comprehension. [\href{../../document/notes/comprehension.pdf}{pdf}]

\noindent Peng Fu. Oct 2013.

\noindent \textbf{Remarks}: A Chapter from my dissertation.


\item A004: Dependently-typed Programming with Scott Encoding. Peng Fu, Aaron Stump. 2013. [\href{../../document/papers/scott-dep.pdf}{pdf}]
 
\item A005: Representing Nonterminating Reductions in F-mu-2. Peng Fu, 2016-2017. [\href{../../document/papers/nonterm.pdf}{pdf}]

\end{itemize}
\section*{Academic Blog}
I try to maintain an academic blog here: \url{https://steprow.wordpress.com/}. 

\end{document}
